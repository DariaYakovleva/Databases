% !TEX encoding = UTF-8
% !TEX program = pdflatex
\documentclass[12pt]{article}

%-------------------------------------
% CS61 students begin copying here
\usepackage{amssymb}

\def\ojoin{\setbox0=\hbox{$\bowtie$}%
  \rule[-.02ex]{.25em}{.4pt}\llap{\rule[\ht0]{.25em}{.4pt}}}
\def\leftouterjoin{\mathbin{\ojoin\mkern-5.8mu\bowtie}}
\def\rightouterjoin{\mathbin{\bowtie\mkern-5.8mu\ojoin}}
\def\fullouterjoin{\mathbin{\ojoin\mkern-5.8mu\bowtie\mkern-5.8mu\ojoin}}

% Relational algebra symbols from ftp://reports.stanford.edu/www/dbgroup_only/latex-macros.html
\newcommand{\select}{\mbox{\Large$\sigma$}}
\newcommand{\cross}{\mbox{$\times$}}
\newcommand{\intersection}{\mbox{$\cap$}}
\newcommand{\intersect}{\mbox{$\cap$}}
\newcommand{\union}{\mbox{$\cup$}}
\newcommand{\join}{\mbox{$\Join$}}
\newcommand{\leftsemijoin}{\mbox{$\mathrel{\raise1pt\hbox{\vrule height5pt
depth0pt width0.6pt\hskip-1.5pt$>$\hskip -2.5pt$<$}}$}}
\newcommand{\rightsemijoin}{\mbox{$\mathrel{\raise1pt\hbox{\hskip-1.5pt$>$\hskip -2.5pt$<$\hskip -1.1pt\vrule height5pt
depth0pt width0.6pt}}$}}
\newcommand{\project}{\mbox{\Large$\pi$}}
\newcommand{\Project}{\mbox{$\Pi$}}
\newcommand{\aggregatefn}{\mbox{\Large$G$}}

% CS61 students end copying here
%-------------------------------------

\usepackage[T2A]{fontenc}
\usepackage[utf8]{inputenc}
\usepackage[russian]{babel}
\usepackage{graphicx}

\usepackage{amsmath}
\textheight=24cm
\textwidth=16cm
\oddsidemargin=0pt
\topmargin=-1.5cm

\title{Домашнее задание 5}
\author{Дарья Яковлева, M3439}
\date{31.10.2016}
\begin{document}
\maketitle
\thispagestyle{empty}

Задание: Составить выражения реляционной алгебры и соответствующие SQL-запросы для базы данных «Деканат».



Отношения
\begin{itemize}

	\item \textbf{Groups}
	
		$G(\underline{GID}, GName)$
	\item \textbf{Students}
	
		$S(\underline{SID}, SName, GID)$
	
	\item \textbf{Lecturers}
	
		$L(\underline{LID}, LName)$

	\item \textbf{Courses}
	
		$C(\underline{CID}, CName)$
		
	\item \textbf{Marks}
	
		$M(Mark, \underline{CID}, \underline{SID})$
	
	\item \textbf{Academin plan}
	
		$P(\underline{LID}, \underline{CID}, \underline{GID})$
		
\end{itemize}

Операции реляционной алгебры и SQL-запросы

\begin{enumerate}	
   \item \textbf{Информацию о студентах с заданной оценкой по предмету «Базы данных»}

   		$ \project_{S.SName, M.Mark} (\select_{C.CName = X } (S \bowtie C \bowtie M)) $
   		
   		 \begin{verbatim}
SELECT students.student_name
	, marks.mark_value
FROM students 
NATURAL JOIN courses
NATURAL JOIN marks 
WHERE courses.course_name = 'Databases';
		\end{verbatim}

   		
    \item \textbf{Информацию о студентах не имеющих оценки по предмету «Базы данных»}

       	\begin{enumerate}	
   			\item \textbf{среди всех студентов}
   			   			
   				$ \project_{S.SName} (S) - \project_{S.SName} (\select_{C.CName = X } (S \bowtie C \bowtie M)) $
   				
				\begin{verbatim}
SELECT students.student_name
FROM students 
EXCEPT
SELECT students.student_name
FROM students 
NATURAL JOIN courses
NATURAL JOIN marks 
WHERE courses.course_name = 'Databases';
				\end{verbatim}
   			
   			\item \textbf{среди студентов, у которых есть этот предмет}
   			
   				$ \project_{S.SName} (\select_{C.CName = X } (S \bowtie P \bowtie C)) - \project_{S.SName} (\select_{C.CName = X } (S \bowtie C \bowtie M)) $
   			
				\begin{verbatim}
SELECT students.student_name
FROM students 
NATURAL JOIN academicplan
NATURAL JOIN courses
WHERE courses.course_name = 'Databases'
EXCEPT
SELECT students.student_name
FROM students 
NATURAL JOIN courses
NATURAL JOIN marks 
WHERE courses.course_name = 'Databases';
				\end{verbatim}
   			
		\end{enumerate}	
       
    
    \item \textbf{Информацию о студентах, имеющих хотя бы одну оценку у заданного лектора}
    
		$ \project_{S.SName} (\select_{L.LName = X } (S \bowtie P \bowtie C \bowtie L \bowtie_{S.SID = M.SID \wedge C.CID = M.CID} M)) $
    
		\begin{verbatim}
SELECT DISTINCT students.student_name
FROM students 
NATURAL JOIN academicplan
NATURAL JOIN courses
NATURAL JOIN lecturers
INNER JOIN marks
	ON students.student_id = marks.student_id AND courses.course_id = marks.course_id
WHERE lecturers.lecturer_name = 'Georgiy Korneev';
		\end{verbatim}
            
    
    \item \textbf{Идентификаторы студентов, не имеющих ни одной оценки у заданного лектора}
    
    		$\project_{S.SName} (S) - \project_{S.SName} (\select_{L.LName = X } (S \bowtie P \bowtie C \bowtie L \bowtie_{S.SID = M.SID \wedge C.CID = M.CID} M)) $
    
		\begin{verbatim}
SELECT students.student_name
FROM students 
EXCEPT
SELECT DISTINCT students.student_name
FROM students 
NATURAL JOIN academicplan
NATURAL JOIN courses
NATURAL JOIN lecturers
INNER JOIN marks
	ON students.student_id = marks.student_id AND courses.course_id = marks.course_id
WHERE lecturers.lecturer_name = 'Georgiy Korneev';
		\end{verbatim}
    
    \item \textbf{Студентов, имеющих оценки по всем предметам заданного лектора}
	    $\project_{S.SName, C.CID}  (S \bowtie P \bowtie M) \div (\project_{C.CID} (\select_{L.LName = X } (P \bowtie L))) $
    
		\begin{verbatim}
SELECT students.student_name
FROM students 
NATURAL JOIN academicplan
NATURAL JOIN courses
NATURAL JOIN lecturers
INNER JOIN marks
	ON students.student_id = marks.student_id AND courses.course_id = marks.course_id
WHERE lecturers.lecturer_name = 'Georgiy Korneev' 
GROUP BY students.student_id
HAVING COUNT(lecturers.lecturer_name) = (
		SELECT COUNT(DISTINCT academicplan.course_id)
		FROM academicplan
		INNER JOIN lecturers
			ON academicplan.lecturer_id = lecturers.lecturer_id
		WHERE lecturers.lecturer_name = 'Georgiy Korneev'
	)
;
		\end{verbatim}
    
    \item \textbf{Для каждого студента имя и предметы, которые он должен посещать}
    
		$ \project_{S.SName, C.CName} (S \bowtie P \bowtie C) $
        
		\begin{verbatim}
SELECT students.student_name
	, courses.course_name
FROM students 
NATURAL JOIN academicplan
NATURAL JOIN courses
ORDER BY students.student_name;
		\end{verbatim}
    
    \item \textbf{По лектору всех студентов, у которых он хоть что-нибудь преподавал}
	
	        $ \project_{S.SName} (\select_{L.LName = X } (S \bowtie P \bowtie L)) $
    
		\begin{verbatim}
SELECT DISTINCT students.student_name
FROM students 
NATURAL JOIN academicplan
NATURAL JOIN lecturers
WHERE lecturers.lecturer_name = 'Georgiy Korneev';
		\end{verbatim}
    
    \item \textbf{Пары студентов, такие, что все сданные первым студентом предметы сдал и второй студент}
    
    		$ (\project_{S.SName, C.CID} (\select_{S.SName = Y, M.Mark \ge 60} (S \bowtie P \bowtie M)) \divideontimes (\project_{S.SName, C.CID} (\select_{S.SName = X, M.Mark \ge 60} (S \bowtie P \bowtie M)))$
    
		\begin{verbatim}

		\end{verbatim}
    
    
    \item \textbf{Такие группы и предметы, что все студенты группы сдали предмет}

       		$ (\project_{S.SID, C.CName} (\select_{M.Mark \ge 60} (S \bowtie P \bowtie C \bowtie M))) \divideontimes (\project_{S.SID, G.GName} (S))$
   
		\begin{verbatim}

		\end{verbatim}
    
    
    \item \textbf{Средний балл студента}
       	\begin{enumerate}	
   			\item \textbf{по идентификатору}   		
				
				$ avg_{M.Mark, \emptyset} (\select_{S.SID = X} (S \bowtie M)) $
				
				\begin{verbatim}
SELECT AVG(marks.mark_value) 
FROM students
NATURAL JOIN marks
WHERE students.student_id = 3;
				\end{verbatim}

   			
   			\item \textbf{для каждого студента}
   			
   				% проблемы с left join
				$ avg_{M.Mark, \{S.SName\}} (S \rightsemijoin M) $   			
								
				\begin{verbatim}
SELECT students.student_name
	, AVG(marks.mark_value) 
FROM students 
LEFT JOIN marks
	ON students.student_id = marks.student_id
GROUP BY students.student_id
ORDER BY students.student_name;
				\end{verbatim}
   			
		\end{enumerate}	
    
    
    \item \textbf{Средний балл средних баллов студентов каждой группы}
    
		$ avg_{M.Mark, \{G.GName\}} (S \bowtie M) $   			    
    
		\begin{verbatim}
SELECT groups.group_name
	, AVG(marks.mark_value) 
FROM groups
NATURAL JOIN students
NATURAL JOIN marks
GROUP BY groups.group_id
ORDER BY groups.group_name;
		\end{verbatim}
    
    
    \item \textbf{Для каждого студента число предметов, которые у него были, число сданных предметов и число не сданных предметов}

		$ (\varepsilon_{Total = count_{C.CID, S.SName} (\project_{C.CID, S.SName} (S \bowtie P))} \circ$
		
		
		$\varepsilon_{Passed = count_{C.CID, S.SName} (\select_{M.Mark \ge 60} (\project_{C.CID, S.SName} (S \bowtie P \bowtie M))} $
		
		$\circ \varepsilon_{Failed=Total - Passed}) (\project_{S.SName} (S))$   			    		
		\begin{verbatim}

		\end{verbatim}


\end{enumerate}


Приложение. База данных.
\begin{verbatim}
CREATE TABLE IF NOT EXISTS groups (
	group_id int PRIMARY KEY,
	group_name varchar(100)
);

CREATE TABLE IF NOT EXISTS students (
	student_id int PRIMARY KEY,
	student_name varchar(100),
	group_id int REFERENCES groups (group_id)
);

CREATE TABLE IF NOT EXISTS lecturers (
	lecturer_id int PRIMARY KEY,
	lecturer_name varchar(50)
);


CREATE TABLE IF NOT EXISTS courses (
	course_id int PRIMARY KEY,
	course_name varchar(50)
);

CREATE TABLE IF NOT EXISTS marks (
	mark_value int,
	course_id int not null REFERENCES courses (course_id),
	student_id int not null REFERENCES students (student_id),
	CHECK (mark_value BETWEEN 0 AND 100),
	PRIMARY KEY (course_id, student_id)
);

CREATE TABLE IF NOT EXISTS academicplan (
	lecturer_id int not null REFERENCES lecturers (lecturer_id),
	course_id int not null REFERENCES courses (course_id),
	group_id int not null REFERENCES groups (group_id),
	PRIMARY KEY (lecturer_id, course_id, group_id)
);
\end{verbatim}

\end{document}

%%% Local Variables:
%%% fill-column: 10
%%% mode: latex
%%% coding: utf-8
%%% TeX-PDF-mode: t
%%% End: