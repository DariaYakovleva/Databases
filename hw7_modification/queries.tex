% !TEX encoding = UTF-8
% !TEX program = pdflatex
\documentclass[12pt]{article}

%-------------------------------------
% CS61 students begin copying here
\usepackage{amssymb}

\def\ojoin{\setbox0=\hbox{$\bowtie$}%
  \rule[-.02ex]{.25em}{.4pt}\llap{\rule[\ht0]{.25em}{.4pt}}}
\def\leftouterjoin{\mathbin{\ojoin\mkern-5.8mu\bowtie}}
\def\rightouterjoin{\mathbin{\bowtie\mkern-5.8mu\ojoin}}
\def\fullouterjoin{\mathbin{\ojoin\mkern-5.8mu\bowtie\mkern-5.8mu\ojoin}}

% Relational algebra symbols from ftp://reports.stanford.edu/www/dbgroup_only/latex-macros.html
\newcommand{\select}{\mbox{\Large$\sigma$}}
\newcommand{\cross}{\mbox{$\times$}}
\newcommand{\intersection}{\mbox{$\cap$}}
\newcommand{\intersect}{\mbox{$\cap$}}
\newcommand{\union}{\mbox{$\cup$}}
\newcommand{\join}{\mbox{$\Join$}}
\newcommand{\leftsemijoin}{\mbox{$\mathrel{\raise1pt\hbox{\vrule height5pt
depth0pt width0.6pt\hskip-1.5pt$>$\hskip -2.5pt$<$}}$}}
\newcommand{\rightsemijoin}{\mbox{$\mathrel{\raise1pt\hbox{\hskip-1.5pt$>$\hskip -2.5pt$<$\hskip -1.1pt\vrule height5pt
depth0pt width0.6pt}}$}}
\newcommand{\project}{\mbox{\Large$\pi$}}
\newcommand{\Project}{\mbox{$\Pi$}}
\newcommand{\aggregatefn}{\mbox{\Large$G$}}

% CS61 students end copying here
%-------------------------------------

\usepackage[T2A]{fontenc}
\usepackage[utf8]{inputenc}
\usepackage[russian]{babel}
% \usepackage{minty}
\usepackage{graphicx}
\usepackage{color} %% это для отображения цвета в коде
\usepackage{listings} %% собственно, это и есть пакет listings

\usepackage{caption}
\DeclareCaptionFont{white}{\color{white}} %% это сделает текст заголовка белым
%% код ниже нарисует серую рамочку вокруг заголовка кода.
\DeclareCaptionFormat{listing}{\colorbox{gray}{\parbox{\textwidth}{#1#2#3}}}
\captionsetup[lstlisting]{format=listing,labelfont=white,textfont=white}

\usepackage{amsmath}
\textheight=24cm
\textwidth=16cm
\oddsidemargin=0pt
\topmargin=-1.5cm

\title{Домашнее задание 7}
\author{Дарья Яковлева, M3439}
\date{22.11.2016}
\begin{document}
\maketitle
\thispagestyle{empty}

Задание: Составить запросы.

~~~

P.S. Будем считать, что у студента долг по предмету, если он изучает этот предмет и имеет по нему менее 60 баллов. 

\begin{enumerate}	
     \item \textbf{Запрос, удаляющий всех студентов, не имеющих долгов}
\begin{lstlisting}     
DELETE
FROM students 
WHERE student_id NOT IN 
(
	SELECT marks.student_id
	FROM marks 
	WHERE marks.mark_value <= 60 and marks.course_id IN
	(
		SELECT academicplan.course_id
		FROM academicplan
		WHERE students.group_id = academicplan.group_id
    )	
);
\end{lstlisting}
     
     \item \textbf{Запрос, удаляющий всех студентов, имеющих 3 и более долгов}
     
     \begin{lstlisting}     
DELETE
FROM students 
WHERE student_id IN 
(
	SELECT marks.student_id
	FROM marks 
	WHERE marks.mark_value <= 60 AND marks.course_id IN
	(
		SELECT academicplan.course_id
		FROM academicplan
		WHERE students.group_id = academicplan.group_id
    )
    GROUP BY marks.student_id
    HAVING COUNT(marks.student_id) >= 3	
);
     
     \end{lstlisting}     
     
     \item \textbf{Запрос, удаляющий все группы, в которых нет студентов}

     \begin{lstlisting}     
DELETE
FROM groups
WHERE group_id NOT IN 
(
	SELECT students.group_id
	FROM students
);    
     \end{lstlisting}     

     \item \textbf{View Losers в котором для каждого студента, имеющего долги указано их количество}
     
     \begin{lstlisting}     
CREATE VIEW Losers AS
SELECT students.student_name, COUNT(marks.student_id)
FROM students
NATURAL JOIN marks
WHERE marks.mark_value <= 60 AND marks.course_id IN
(
	SELECT academicplan.course_id
	FROM academicplan
	WHERE students.group_id = academicplan.group_id
)	
GROUP BY students.student_id;
     
     \end{lstlisting}     
     
     \item \textbf{Таблица LoserT, в которой содержится та же информация, что во view Losers. Эта таблица должна автоматически обновляться при изменении таблицы с баллами}

     \begin{lstlisting}     
CREATE TABLE LoserT AS SELECT * FROM Losers;

CREATE OR REPLACE FUNCTION loser_procedure() 
RETURNS TRIGGER AS $LoserTrig$
    BEGIN
		DELETE FROM LoserT *;
		INSERT INTO LoserT (SELECT * FROM Losers);
		RETURN NEW;
    END;
$LoserTrig$ LANGUAGE plpgsql;


CREATE TRIGGER LoserTrig
AFTER INSERT OR UPDATE OR DELETE ON marks
FOR EACH ROW EXECUTE PROCEDURE loser_procedure();
     
     \end{lstlisting}     

     \item \textbf{Отключить автоматическое обновление таблицы LoserT}
     
     \begin{lstlisting}     
     
DROP TRIGGER IF EXISTS LoserTrig ON marks;
     
     \end{lstlisting}     
     
     \item \textbf{Запрос (один), который обновляет таблицу LoserT, используя данные из таблицы NewPoints, в которой содержится информация о баллах, проставленных за последний день}
     
     \begin{lstlisting}     
     
     \end{lstlisting}     
     
     \item \textbf{Проверка того, что все студенты одной группы изучают один и тот же набор курсов}

	
     \begin{lstlisting}     
     
     \end{lstlisting}     

     \item \textbf{Триггер, не позволяющий уменьшить баллы студента по предмету. При попытке такого изменения, баллы изменяться не должны} 		                         		               		                                        		               		                         		               		     	
     \begin{lstlisting}     

CREATE OR REPLACE FUNCTION marks_procedure() 
 RETURNS TRIGGER AS $LoserTrig$
    BEGIN
    	IF (NEW.mark_value < OLD.mark_value) THEN 
	    	NEW.mark_value = OLD.mark_value;
    		RETURN NEW;
    	END IF;
    	RETURN NULL;
    END;
$LoserTrig$ LANGUAGE plpgsql;


CREATE TRIGGER marksUpdate
BEFORE UPDATE ON marks
FOR EACH ROW EXECUTE PROCEDURE marks_procedure();     
     
     \end{lstlisting}     

                        
%	\begin{minty}

%	\end{minty}
        
\end{enumerate}    



\textbf{Приложение. База данных.}
\begin{lstlisting}     
CREATE TABLE IF NOT EXISTS groups (
	group_id int PRIMARY KEY,
	group_name varchar(100)
);

CREATE TABLE IF NOT EXISTS students (
	student_id int PRIMARY KEY,
	student_name varchar(100),
	group_id int REFERENCES groups (group_id) 
		ON DELETE CASCADE ON UPDATE CASCADE
);

CREATE TABLE IF NOT EXISTS lecturers (
	lecturer_id int PRIMARY KEY,
	lecturer_name varchar(50)
);


CREATE TABLE IF NOT EXISTS courses (
	course_id int PRIMARY KEY,
	course_name varchar(50)
);

CREATE TABLE IF NOT EXISTS marks (
	mark_value int,
	course_id int not null REFERENCES courses (course_id) 
		ON DELETE CASCADE ON UPDATE CASCADE,
	student_id int not null REFERENCES students (student_id) 
		ON DELETE CASCADE ON UPDATE CASCADE,
	CHECK (mark_value BETWEEN 0 AND 100),
	PRIMARY KEY (course_id, student_id)
);

CREATE TABLE IF NOT EXISTS academicplan (
	lecturer_id int not null REFERENCES lecturers (lecturer_id) 
		ON DELETE CASCADE ON UPDATE CASCADE,
	course_id int not null REFERENCES courses (course_id) 
		ON DELETE CASCADE ON UPDATE CASCADE,
	group_id int not null REFERENCES groups (group_id)
		 ON DELETE CASCADE ON UPDATE CASCADE,
	PRIMARY KEY (lecturer_id, course_id, group_id)
);
\end{lstlisting}     

\end{document}

%%% Local Variables:
%%% fill-column: 10
%%% mode: latex
%%% coding: utf-8
%%% TeX-PDF-mode: t
%%% End: