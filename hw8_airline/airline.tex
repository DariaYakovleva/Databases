% !TEX encoding = UTF-8
% !TEX program = pdflatex
\documentclass[12pt]{article}

%-------------------------------------
% CS61 students begin copying here
\usepackage{amssymb}

\def\ojoin{\setbox0=\hbox{$\bowtie$}%
  \rule[-.02ex]{.25em}{.4pt}\llap{\rule[\ht0]{.25em}{.4pt}}}
\def\leftouterjoin{\mathbin{\ojoin\mkern-5.8mu\bowtie}}
\def\rightouterjoin{\mathbin{\bowtie\mkern-5.8mu\ojoin}}
\def\fullouterjoin{\mathbin{\ojoin\mkern-5.8mu\bowtie\mkern-5.8mu\ojoin}}

% Relational algebra symbols from ftp://reports.stanford.edu/www/dbgroup_only/latex-macros.html
\newcommand{\select}{\mbox{\Large$\sigma$}}
\newcommand{\cross}{\mbox{$\times$}}
\newcommand{\intersection}{\mbox{$\cap$}}
\newcommand{\intersect}{\mbox{$\cap$}}
\newcommand{\union}{\mbox{$\cup$}}
\newcommand{\join}{\mbox{$\Join$}}
\newcommand{\leftsemijoin}{\mbox{$\mathrel{\raise1pt\hbox{\vrule height5pt
depth0pt width0.6pt\hskip-1.5pt$>$\hskip -2.5pt$<$}}$}}
\newcommand{\rightsemijoin}{\mbox{$\mathrel{\raise1pt\hbox{\hskip-1.5pt$>$\hskip -2.5pt$<$\hskip -1.1pt\vrule height5pt
depth0pt width0.6pt}}$}}
\newcommand{\project}{\mbox{\Large$\pi$}}
\newcommand{\Project}{\mbox{$\Pi$}}
\newcommand{\aggregatefn}{\mbox{\Large$G$}}

% CS61 students end copying here
%-------------------------------------

\usepackage[T2A]{fontenc}
\usepackage[utf8]{inputenc}
\usepackage[russian]{babel}
% \usepackage{minty}
\usepackage{graphicx}
\usepackage{color} %% это для отображения цвета в коде
\usepackage{listings} %% собственно, это и есть пакет listings

\usepackage{caption}
\DeclareCaptionFont{white}{\color{white}} %% это сделает текст заголовка белым
%% код ниже нарисует серую рамочку вокруг заголовка кода.
\DeclareCaptionFormat{listing}{\colorbox{gray}{\parbox{\textwidth}{#1#2#3}}}
\captionsetup[lstlisting]{format=listing,labelfont=white,textfont=white}

\usepackage{amsmath}
\textheight=24cm
\textwidth=16cm
\oddsidemargin=0pt
\topmargin=-1.5cm

\title{Домашнее задание 8}
\author{Дарья Яковлева, M3439}
\date{29.11.2016}
\begin{document}
\maketitle
\thispagestyle{empty}

Задание: Реализовать базу данных Airline, которая содержит информацию о рейсах самолётов.

\begin{enumerate}	
 \item \textbf{База данных}
 
\begin{lstlisting}     

CREATE DATABASE airline;

CREATE TABLE IF NOT EXISTS planes (
	plane_id int PRIMARY KEY,
	capacity int not null
);

CREATE TABLE IF NOT EXISTS flights (
	flight_id SERIAL PRIMARY KEY,
	flight_num int,
	flight_time timestamp without time zone,
	plane_id int not null REFERENCES planes (plane_id) 
	ON DELETE CASCADE ON UPDATE CASCADE
);

CREATE TABLE IF NOT EXISTS seats (
	flight_id int REFERENCES flights (flight_id) 
	ON DELETE CASCADE ON UPDATE CASCADE,
	seat_no int not null,
	status int not null, --  1 -- reserved, 2 -- bought
	reserved_time timestamp without time zone,
	PRIMARY KEY(flight_id, seat_no)
);

INSERT INTO planes (plane_id, capacity)
	VALUES (1, 100),
		   (2, 120),
		   (3, 50);

INSERT INTO flights (flight_num, flight_time, plane_id)
	VALUES (1, timestamp '2016-11-28 20:38:40', 1),
		   (2, timestamp '2016-11-29 19:38:40', 2),
		   (3, timestamp '2016-12-01 01:38:20', 1),
		   (4, timestamp '2016-12-14 03:38:40', 2),
		   (5, timestamp '2016-12-16 13:38:10', 2);

INSERT INTO seats (flight_id, seat_no, status, reserved_time)
	VALUES (1, 1, 1, timestamp '2016-11-28 13:38:10'),
		   (1, 2, 2, timestamp '2016-11-28 10:38:10'),
		   (1, 3, 1, timestamp '2016-11-27 13:38:10'),
		   (2, 2, 2, timestamp '2016-11-26 13:38:10'),
		   (2, 5, 1, timestamp '2016-11-29 13:38:10');
\end{lstlisting}     

     \item \textbf{Поддержка бронирования и покупки мест}

\begin{lstlisting} 

CREATE OR REPLACE FUNCTION seats_action() RETURNS TRIGGER AS $SeatsAct$
	DECLARE
		timespent interval;
		cur_bought int;
		flight_time interval;
		cur_capacity int;		
    BEGIN
    	timespent := NEW.reserved_time - (SELECT flight_time 
    	FROM flights WHERE NEW.flights_id = flights.flights_id);
    	flight_time := now() - (SELECT flight_time FROM flights 
    	WHERE NEW.flights_id = flights.flights_id);
    	cur_capacity := SELECT capacity FROM planes 
    	WHERE plane_id = (
	    	SELECT plane_id FROM flights 
	    	WHERE flight_id = NEW.flight_id);

	    IF (NEW.seat_no > cur_capacity) 
	    THEN
	    	NEW := OLD;
	    	RETURN NULL;
	    END IF;
    	
    	
    	IF (NEW.status = 2 
    		AND 
    		(SELECT EXTRACT (DAY FROM timespent)) = 0 
    		AND 
    		(SELECT EXTRACT (HOUR FROM timespent)) < 2) 
    		OR 
    		(NEW.status = 1 
    		AND 
    		(SELECT EXTRACT (DAY FROM timespent)) < 1
    		AND (SELECT EXTRACT (DAY FROM flight_time)) > 0) 
    	THEN
    		NEW := OLD;
    		RETURN OLD;
    	END IF;  
    	
    	cur_bought := (SELECT count(seats_no) 
    	FROM seats WHERE flight_id = NEW.flight_id);

    	IF cur_bought = (SELECT capacity FROM planes WHERE 
    	plane_id = (
	    	SELECT plane_id FROM flights
	    	 WHERE flight_id = NEW.flight_id))
	    THEN
	    	NEW := OLD;
	    	RETURN NULL;
	    END IF; 		


        IF (TG_OP = 'UPDATE') AND (NEW.status = 1) AND (OLD.status = 1)
        THEN
            NEW.reserved_time := now();
            RETURN NEW;
        ELSIF (TG_OP = 'INSERT') 
        THEN
			IF (
				SELECT count(*) 
	    		FROM seats
		    	WHERE flight_id = NEW.flight_id AND
    				  seat_no = NEW.seat_no) > 0 
    			AND (
    			SELECT status
	    		FROM seats
		    	WHERE flight_id = NEW.flight_id AND
    				  seat_no = NEW.seat_no) = 1
    		THEN
    			IF (now() - (SELECT reserved_time
		    		FROM seats
	    			WHERE flight_id = NEW.flight_id AND
	    			  seat_no = NEW.seat_no)) > 24
	    		THEN
	    			DELETE 
		    		FROM seats       	
	    			WHERE flight_id = NEW.flight_id AND
    			  		seat_no = NEW.seat_no;
    			  	RETURN NEW;
    			ELSE
    				NEW := OLD;
    				RETURN NULL;
    			END IF;
    		END IF;        			
    	END IF;
    	RETURN NEW;
    END;
$SeatsAct$ LANGUAGE plpgsql;


CREATE TRIGGER SeatsAct
BEFORE INSERT OR UPDATE ON seats
FOR EACH ROW EXECUTE PROCEDURE seats_action();

\end{lstlisting}

          \item \textbf{Добавление индексов}
          
          	Нужно добавить индекс $(flight\_id, flight\_time)$, так как это это большой и частый запрос.
          	
          	Не нужно добавлять индексы $(flight\_id, seat\_no)$, $(flight\_id, reserved\_time)$, так как это частые запросы, но они ограничены размером самолета, который редко бывает больше 1000.
          	
          \item \textbf{Пусть частым запросом является определение средней заполненности самолёта по рейсу. Какие индексы могут помочь при исполнении данного запроса?}
          
          В этом случае индекс $(flight\_id, status)$ может помочь, так с помощью него можно посчитать среднюю заполненость самолета.
          
          \item \textbf{Добавление индексов на языке SQL}
          \begin{lstlisting} 
          CREATE INDEX USING HASH ON TABLE
           flights (flight_id, flight_time)
          CREATE INDEX USING HASH ON TABLE 
          seats (flight_id, status)
          \end{lstlisting} 
          

\end{document}

%%% Local Variables:
%%% fill-column: 10
%%% mode: latex
%%% coding: utf-8
%%% TeX-PDF-mode: t
%%% End: