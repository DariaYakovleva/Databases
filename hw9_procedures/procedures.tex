% !TEX encoding = UTF-8
% !TEX program = pdflatex
\documentclass[12pt]{article}

%-------------------------------------
% CS61 students begin copying here
\usepackage{amssymb}

\def\ojoin{\setbox0=\hbox{$\bowtie$}%
  \rule[-.02ex]{.25em}{.4pt}\llap{\rule[\ht0]{.25em}{.4pt}}}
\def\leftouterjoin{\mathbin{\ojoin\mkern-5.8mu\bowtie}}
\def\rightouterjoin{\mathbin{\bowtie\mkern-5.8mu\ojoin}}
\def\fullouterjoin{\mathbin{\ojoin\mkern-5.8mu\bowtie\mkern-5.8mu\ojoin}}

% Relational algebra symbols from ftp://reports.stanford.edu/www/dbgroup_only/latex-macros.html
\newcommand{\select}{\mbox{\Large$\sigma$}}
\newcommand{\cross}{\mbox{$\times$}}
\newcommand{\intersection}{\mbox{$\cap$}}
\newcommand{\intersect}{\mbox{$\cap$}}
\newcommand{\union}{\mbox{$\cup$}}
\newcommand{\join}{\mbox{$\Join$}}
\newcommand{\leftsemijoin}{\mbox{$\mathrel{\raise1pt\hbox{\vrule height5pt
depth0pt width0.6pt\hskip-1.5pt$>$\hskip -2.5pt$<$}}$}}
\newcommand{\rightsemijoin}{\mbox{$\mathrel{\raise1pt\hbox{\hskip-1.5pt$>$\hskip -2.5pt$<$\hskip -1.1pt\vrule height5pt
depth0pt width0.6pt}}$}}
\newcommand{\project}{\mbox{\Large$\pi$}}
\newcommand{\Project}{\mbox{$\Pi$}}
\newcommand{\aggregatefn}{\mbox{\Large$G$}}

% CS61 students end copying here
%-------------------------------------

\usepackage[T2A]{fontenc}
\usepackage[utf8]{inputenc}
\usepackage[russian]{babel}
% \usepackage{minty}
\usepackage{graphicx}
\usepackage{color} %% это для отображения цвета в коде
\usepackage{listings} %% собственно, это и есть пакет listings

\usepackage{caption}
\DeclareCaptionFont{white}{\color{white}} %% это сделает текст заголовка белым
%% код ниже нарисует серую рамочку вокруг заголовка кода.
\DeclareCaptionFormat{listing}{\colorbox{gray}{\parbox{\textwidth}{#1#2#3}}}
\captionsetup[lstlisting]{format=listing,labelfont=white,textfont=white}

\usepackage{amsmath}
\textheight=24cm
\textwidth=16cm
\oddsidemargin=0pt
\topmargin=-1.5cm

\title{Домашнее задание 9}
\author{Дарья Яковлева, M3439}
\date{06.12.2016}
\begin{document}
\maketitle
\thispagestyle{empty}

Задание:  Реализовать запросы к базе данных Airline с применением хранимых процедур и функций.

Будем считать, что триггер из предыдущего задания работает корректно.

\begin{enumerate}	

          \item \textbf{ FreeSeats(FlightId) — список мест, доступных для продажи и бронирования.}

          \begin{lstlisting} 
DROP FUNCTION free_seats(fid int);
CREATE OR REPLACE FUNCTION free_seats(fid int) RETURNS 
TABLE(seat int) AS $$
	DECLARE
		cap int;
	BEGIN
    	cap := (SELECT capacity FROM planes WHERE plane_id = (
	    	SELECT plane_id FROM flights WHERE flight_id = fid));

	    RETURN QUERY (SELECT * FROM generate_series(1, cap) 
	    AS X WHERE X NOT IN 
    		(SELECT seat_no FROM seats WHERE flight_id = fid));

    END;
$$ LANGUAGE plpgsql;

SELECT * FROM free_seats(1);
SELECT * FROM free_seats(2);

           \end{lstlisting} 

          \item \textbf{ Reserve(FlightId, SeatNo) — пытается забронировать место. Возвращает истину, если удалось и ложь — в противном случае.}

          \begin{lstlisting} 
CREATE OR REPLACE FUNCTION reserve(fid int, snum int) RETURNS BOOLEAN AS $$
    BEGIN
	   	INSERT INTO seats (flight_id, seat_no, 
	   	status, reserved_time)
				VALUES (fid, snum, 1, now());
		RETURN FOUND;
    END;
$$ LANGUAGE plpgsql;

SELECT reserve(1, 1);
           \end{lstlisting} 
                    
          \item \textbf{ExtendReservation(FlightId, SeatNo) — пытается продлить бронь места. Возвращает истину, если удалось и ложь — в противном случае.}

          \begin{lstlisting} 
CREATE OR REPLACE FUNCTION extend_reservation(fid int, snum int)
 RETURNS BOOLEAN AS $$
    BEGIN
		UPDATE seats SET reserved_time = now()
		 WHERE flight_id = fid AND seat_no = snum;
	    RETURN FOUND;
    END;
$$ LANGUAGE plpgsql;

SELECT extend_reservation(1, 1);
           \end{lstlisting} 

          \item \textbf{ BuyFree(FlightId, SeatNo) — пытается купить свободное место. Возвращает истину, если удалось и ложь — в противном случае. }

          \begin{lstlisting} 
CREATE OR REPLACE FUNCTION buy_free(fid int, snum int)
 RETURNS BOOLEAN AS $$
    BEGIN
		INSERT INTO seats (flight_id, seat_no, 
		status, reserved_time)
			VALUES (fid, snum, 2, now());
		RETURN FOUND;		    	
    END;
$$ LANGUAGE plpgsql;

SELECT buy_free(1, 1);

           \end{lstlisting} 

          \item \textbf{ BuyReserved(FlightId, SeatNo) — пытается выкупить забронированное место. Возвращает истину, если удалось и ложь — в противном случае. }

          \begin{lstlisting} 
CREATE OR REPLACE FUNCTION buy_reserved(fid int, snum int) 
RETURNS BOOLEAN AS $$
    BEGIN
		UPDATE seats SET status = 2 WHERE flight_id = fid 
		AND seat_no = snum;
	    RETURN FOUND;
    END;
$$ LANGUAGE plpgsql;

SELECT buy_reserved(1, 1);

           \end{lstlisting} 



          \item \textbf{FlightStatistics() — возвращает статистику по рейсам: возможность бронирования и покупки, число свободных, забронированных и проданных мест.}

          \begin{lstlisting} 


           \end{lstlisting} 

          \item \textbf{CompressSeats(FlightId) — оптимизирует занятость мест в самолете. В результате оптимизации, в начале самолета должны быть купленные места, затем — забронированные, а в конце — свободные. Примечание: клиенты, которые уже выкупили билеты так же должны быть пересажены. }

          \begin{lstlisting} 


           \end{lstlisting} 

        
\end{document}

%%% Local Variables:
%%% fill-column: 10
%%% mode: latex
%%% coding: utf-8
%%% TeX-PDF-mode: t
%%% End: